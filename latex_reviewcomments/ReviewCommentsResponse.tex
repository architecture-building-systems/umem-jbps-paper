\documentclass[answers,12pt]{exam}
\usepackage{xcolor}
\definecolor{SolutionColor}{rgb}{0.1,0.3,1}
\usepackage{lipsum}

\renewcommand{\thequestion}{\alph{question} }
\renewcommand\questionlabel{\llap{\thequestion)}}

%\pointsinrightmargin
%\boxedpoints
\unframedsolutions
\shadedsolutions
\definecolor{SolutionColor}{rgb}{0.9,0.9,1}
\renewcommand{\solutiontitle}{}

\usepackage{titling}
\newcommand{\subtitle}[1]{%
  \posttitle{%
    \par\end{center}
    \begin{center}\large#1\end{center}
    \vskip0.5em}%
}

% \setlength{\droptitle}{-10em}   % This is your set screw

\title{Review Comment Responses to Journal of Building Performance Simulation Submission}
\subtitle{Paper Title: Urban and building multiscale co-simulation: case study implementations on two university campuses 
}
% \email{miller.clayton@arch.ethz.ch}
\author{Clayton Miller, Daren Thomas, J\'er\^ome K\"ampf, Arno Schlueter}

% \subtitle{ETH Z\"urich, Institute of Technology in Architecture (ITA), Architecture and Sustainable Building Technologies (SuAT)}
% \subtitle{*Corresponding author: Phone: +1-402-403-0090, miller.clayton@arch.ethz.ch}

\begin{document}


\maketitle


% \textcolor{red}{Red text is internal notes of things to be done}

% Black text shaded in blue is the formal responses to the reviewers

\section{Review 1}

\subsection{General comments from reviewer}
This is a valuable paper that pinpoints the differences in estimated energy consumption between simulations of solo building and buildings modeled in an urban context.  The paper relies on am emerging simulation workflow and extends its application to two real-building scenarios.  Work by the authors and others is amply referenced; together the text and the references provide excellent explanation of the linkages between building and city simulations. 

The paper can be improved in a large number of small ways, as indicated in specific comments.  In particular, the central figures. 10, 11, 14 and 15 and accompanying text need attention to improve clarity and correct inconsistencies between text and images.  The suggested improvements should be straightforward. 

\subsection{Response to general comments}
\begin{solution}
Thank you for the comments on the importance and value of the paper and we welcome the comments and feedback to be covered in this review.
\end{solution}

\subsection{Revision items}
\begin{questions}

\question 1. p. 1.  Abstract.  The abstract quantifies percentage “impacts” of co-simulation of heating and cooling consumption but does not establish a reference: percentage change relative to what base case?  The abstract promises a discussion of “strengths and weaknesses of the developed process” but the paper does not include a discussion section and the conclusion does not summarize strengths and weaknesses. 
\begin{solution}
\textcolor{red}{Finish discussion section}

The abstract has been updated to reflect that the baseline of these comparison metrics are the EnergyPlus solo simulations of each of the engines.

A discussion section has been added to address the strengths and weaknesses mentioned in the abstract -- advantages are related to the enhancement of each of the 
engines and disadvantages related to the increase in computing power needed to undertake the process.

% \emph{The majority of these changes can be found in lines X-X.}
\end{solution}

\question  2. p. 1. Introduction. Lines 3-5 state that “2.5 billion people are expected to join cities throughout the world.” What is the associated time frame – i.e., by what year? 
\begin{solution}
\textcolor{red}{Confirm the following is true:} The United Nations World Urbanization prospectus projected that figure by 2030 -- this fact has been updated in the introduction
\end{solution}

\question 3. p. 1, line 13. Please consider a lower-case “S” for solver.   
\begin{solution}
Lower-case "S" has been used.
\end{solution}

\question 4. p. 1, lines 21-23.  The text states that a single equation approximates the total mass flow rate required to meet the sensible and later loads of each building.”  In this reviewer’s experience, there are separate mass-conservation equations for air and water vapor as well as an energy-balance equation.  Perhaps the authors could succinctly amplify their statement. 
\begin{solution}
\textcolor{red}{Jerome -- is this statement regarding CitySim correct? If so, how can we justify it with respect to this comment?} 
\end{solution}

\question 5. p. 2, line 40.  Please replace “phenomenon” with the plural “phenomena.” 
\begin{solution}
Correction made.
\end{solution}

\question 6. p. 2, line 60.  The text appears to limit the second case study to CitySim simulation output but the abstract and, later, line 224, include analysis of EnergyPlus output. 
\begin{solution}
The reference to the EnergyPlus simulation for this case study was removed.
\end{solution}

\question 7. p. 3.  Figure 1 includes acronyms not defined in the text or caption. 
\begin{solution}
{red}{Does anyone know what COSMO-CMM, UMC Model, and BIS stand for?} 
Explanation of the acronyms is now included in the text.
\end{solution}

\question 8. p. 3, lines 62-68.  For grammatical consistency, the achievements should all start with verbs (automate, co-simulate, implement) or nouns (automation, co-simulation, implementation).  The third achievement should specify what simulation model is being calibrated. 
\begin{solution}
Points were revised to all be verbs and the second scenario was indicated as the model being calibrated.
\end{solution}

\question 9. p. 3, lines 88-89.  Please note that Bueno et al. 2013 generated urban weather with a nodal model and did not directly couple EnergyPlus to a nodal representation of the urban canopy layer.  The accuracy of this approach depends on how faithfully users and the model represent the buildings that will later be simulated in detail in EnergyPlus with the modified weather file. 
\begin{solution}
It was noted that this study doesn't directly couple the simulation engines, but only connects them through a modified weather file.
\end{solution}

\question 10. p. 4, line 97.  Please consider “earliest design possible and it can be used…” or “earliest design possible that can be used….” 
\begin{solution}
Change made to the latter suggestion
\end{solution}

\question 11. p. 4, Figure 2. The test explains the paths associated with the dotted vertical lines and solid lines.  Presumably the use of the latter is intended to show the co-simulation path but the use of the dashed line in the related Figure 3 is not clear.  Further, Figure 3 is quite small; at minimum, the CItySIm block with its italicized vertical text, fmilib, could be modestly larger.  
\begin{solution}
\textcolor{red}{Daren -- what was our logic behind the dotted lines?}
\end{solution}

\question 12. p. 5, lines 145-154.  The text states that “several key urban-scale weather variables are send to EnergyPlus from CitySim” but does not succinctly state how CItySIm calculates these variables.  The text might indicate why the authors consider the CitySim occupancy model to be more robust.  Finally, as a minor point of English grammar, please consider removing the comma after “urban environments” in line 150 or replacing “that” with “which.” 
\begin{solution}
\textcolor{red}{Jerome -- can you explain quickly in the text how CitySim calculated these variables and why its better than using EnergyPlus's weather file inputs?}
\end{solution}

\question 13. p. 6.  The text on this page (or, apparently, any other) does not refer to Tables 1 and 2. 
\begin{solution}
References for Table 1 and 2 are no included in the text pertaining to those variables.
\end{solution}

\question 14. p. 7, line 178.  Equation 2 lacks a view factor, as is included in Evins et al., Equation 9 and in the EnergyPlus Engineering Reference. 
\begin{solution}
\textcolor{red}{Jerome and Daren - can you confirm the accuracy of the equations?}
\end{solution}

\question 15. p. 7, lines 195-196.  The sentence beginning with “An automated workflow” lacks a verb.  Please consider replacing “to tie” with “ties.” 
\begin{solution}
Correction made.
\end{solution}

\question 16. p. 9, line 215.  Please consider replacing “compare” with “quantify,” given that the sentence does not establish the basis of comparison (compare what with what)). 
\begin{solution}
Correction made.
\end{solution}

\question 17. p. 10. Section 3.2.  The text states that the campus “geometry was converted into an EnergyPlus input file as seen in Figure 7.  However, this figure does not clearly indicate what geometry was included in the IDF and the caption states that the HPI building was modeled in Open Studio.   
\begin{solution}
The caption for Figure 7 has been updated to clarify the situation in which the HPI building was extracted from the CiytSim model, however it is illustrated using
OpenStudio due to that software's rendering capabilities. The caption also clearly states which building is being simulationed.
\end{solution}

\question 18. p. 10, lines 246-248.  Please consider replacing “are screened” with “were screened” and removing the hyphen after plug. 
\begin{solution}
Correction made.
\end{solution}

\question 19. p. 10, section 3.3.  Either the text (lines 245-252) or the caption for Figure 8 should say a bit more (a sentence would be adequate) about the clustering process.  For example, the caption could note that the rows for each month refer to days of the week (if this is true) and could name (describe) each of the clusters (i.e., clusters that nominally align with weekdays, Saturdays and Sundays/school holidays). 
\begin{solution}
The caption for Figure 8 was revised significantly to describe what the reader is seeing in the typical average profiles on the left and the days corresponding with
each cluster on the right. Text was added in the preceding paragraphs that briefly describe the significance of each of the clusters.
\end{solution}

\question 20. p. 10., lines 253-259.  Please rewrite the sentence to avoid “compared … to create… scenario for comparison.” What is meant by “measurements were deemed most accurate during the appropriate seasons?”
\begin{solution}
The first sentence of that paragraph was rewritten to avoid the use of "comared" and "create a scenario for comparison". The sentence pertaining to the seasonal nature of the data analysis was removed as it didn't add any information to the fact that only heating and cooling seasonal data was used for each calibration.
\end{solution}

\question 21. p. 11, lines 261-262.  The NMBE for heating and cooling are exactly the same and the NMBE for cooling is not in accord with the bar charts in Figure 9. 
\begin{solution}
\end{solution}

\question 22. p. 11, section 3.4.  Please see the following comments for the two associated figures.  In addition, the presentation of results from the study of Miller et al. is not clear.  Does the Miller study compare solo and co-simulation results?  If so, which results are the reference for the percentage changes? Should the reader conclude that the differences between Miller and this study are comparable (which is comforting) or that deviations are significant, in which case they should be explained. 
\begin{solution}
\textcolor{red}{Confirm in the Juypter notebook}
\end{solution}

\question 24. p. 12, Figure 11.  The text states Figure 11 shows differences in predicted heating consumption and notes there is a consistent offset across days in January.  Figure 11 shows hourly variations for presumably averaged days and single points for each month; in short, variations across days cannot be discerned.  Are these variations important to the comparison?  The order of the bars matches Figure 10 but disagrees with Figure 9.  The text compares co-simulation with measurement (using the latter as the reference, which seems appropriate) and shows 2\% annual difference but the numbers over the bars appear to compare co-simulation and measurement with solo simulation and show 6.5\% difference.  As in Figure 11, the numbers (not percentages) over the bars are not explained. 
\begin{solution}
\textcolor{red}{Update graphics}
\end{solution}

\question 25. p. 12, section 3.5.  The text should include a reference to the Minergie standard and the caption or image in Figure 12 should identify the STCC building. 
\begin{solution}
\textcolor{red}{Add Minergie as a reference}
\end{solution}

\question 
26. p. 13, line 308.  Please explain SIA and/or provide an appropriate reference.   
\begin{solution}
\textcolor{red}{Add SIA and explain}
\end{solution}

\question 
27. p. 13, lines 310-313.  The text states that the CItySIm model was created from a DXF file from EnergyPlus.  Was this export used solely for the STCC building or was the remainder of the Quartier Nord as shown in Figure 13 designed in Open Studio? 
\begin{solution}
\textcolor{red}{Jerome -- any feedback on this?}
\end{solution}

\question 
28. P. 13, lines 316-322.  The reference in the text for percentage changes is the EnergyPlus simulation but Figures 14 and 15 place the percentages (with opposite sign) about the solo simulations.  The text states that discrepancies between solo and co-simulations are due to differences in calculation of heating and cooling loads.  Please elaborate. 
\begin{solution}
\textcolor{red}{Confirm numbers in graphics}
\end{solution}

\question 
29. P. 14, section 3.8.  The text states that the Quartier Nord building was also simulated. This is a bit confusing because Quartier Nord previously referred to the complex.  Does the building refer to the building on the right in Figures 12 and 13 – i.e., everything except the STCC?  The text also states that differences between solo and co-simulations are small because the surroundings are sparse.  If this is the case, would one expect better agreement between solo and co-simulations for the neighboring STCC?  Also, please consider replacing “due to” with “because” in line 327. 
\begin{solution}
\textcolor{red}{Jerome -- can you clarify this?}
\end{solution}

\question 
30. p. 15.  Conclusion.  There is no summary of strengths and weakness of the process presented in the paper and the future work is limited only to more practical implementation.  Is there nothing to improve? 
\begin{solution}
\textcolor{red}{Update with strengths and weaknesses in the discussion}
\end{solution}

\question 
31. p. 16. The first and last references, by Miller et al. and Zhang et al., lack volume and page numbers.  Throughout the reference section, the authors should consider consistency in use of initials versus given names.   
\begin{solution}
\textcolor{red}{Update references}
\end{solution}

\pagebreak
\section{Review 2}

\subsection{General comments from reviewer}
This is a nice study in a series of studies that use institutional buildings for demonstration of different co-simulation strategies for modeling of urban environments and building energy consumption. Here are several review comments that are mostly focused on the fact that actual execution of co-simulation was not discussed in details; the study mostly presented the co-simulation workflow and results. 


\subsection{Response to general comments}
\begin{solution}
Thank you for taking time to perform this review and for the in-depth comments related to discussing the execution of the actual co-simulation.
\end{solution}

\subsection{Revision items}

\question 
(1) At the top of page 2, the paper describes advantages of EnergyPlus. However, it is not clear why this simulation engine is used when several others could perform the presented analyses. It is important to know what technical aspects of the co-simulation procedure specifically drove the authors to use EnergyPlus. 
\begin{solution}
\textcolor{red}{Update references}

\end{solution}

\question 
(2) The actual co-simulation, which is the main contribution of this study is not well discussed. Specifically, the paper needs to describe: (a) frequency of data exchanges, (b) whether any data averaging took place, (c) whether the numerical scheme is stable and grid independent, (d) what is the time step for each engine?, (e) what is the simulation time?, (f) are these annual, monthly, or daily simulations?, (g) what convergence criterion was used? 
\begin{solution}
\textcolor{red}{Daren -- can you take a stab at these?}
a) \\
b) \\
c) \\
d) \\
e) \\
f) \\
g) \\
\end{solution}

\question 
(3) Figure 3, Section 2.3, and Figure 4 are the same or similar to author’s previous publication. A citation would be sufficient. 
\begin{solution}
We chose to replicate these figures in this publication to clarify the bigger picture of the scope of the study. We believe these figures help to put the current study in context in a better way.
\end{solution}

\question 
(4) In Section 3.3, the paper needs many more details on the calibration process, including the specification of all calibration steps undertaken, rather than just cursory listing some of the steps. 
\begin{solution}
\textcolor{red}{Explain the calibration process in much more detail}
\end{solution}

\question 
(5) Please, discuss results in Figure 10. What are the numerical or physical reasons for the “solo” case to underperform? 
\begin{solution}
\textcolor{red}{Why are there differences?}

\end{solution}

\question 
(6) The conclusions need content on co-simulation execution challenges and how they were overcome during the course of that this study.
\begin{solution}
\textcolor{red}{Strenghts and weaknesses and advantages/disadvantages in the}

\end{solution}
\end{questions}
\end{document}